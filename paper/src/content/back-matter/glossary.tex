\textbf{aiman:} An AI-assisted command-line interface tool developed for this research that provides intelligent error correction and command suggestions to improve CLI usability.

\textbf{Artificial Intelligence (AI):} The simulation of human intelligence processes by machines, especially computer systems, used in this context to enhance command-line interfaces.

\textbf{Automation:} The use of scripts and batch processing to execute tasks automatically without manual intervention, a key benefit of CLI systems.

\textbf{Bash:} A Unix shell and command language that serves as the default command-line interpreter in many Linux distributions and macOS systems.

\textbf{Command-Line Interface (CLI):} A text-based user interface used to interact with computer programs or operating systems by typing commands directly.

\textbf{Contextual Assistance:} AI-powered help that adapts suggestions based on the user's current working environment, file structure, and command history.

\textbf{Counterbalancing:} An experimental design technique used to control for order effects by varying the sequence in which participants experience different conditions.

\textbf{Docker:} A containerization platform used in this research to create consistent, isolated testing environments for CLI experiments.

\textbf{Error Correction:} The process of detecting and fixing mistakes in command input or execution, enhanced by AI in modern CLI tools.

\textbf{Exit Code:} A numeric value returned by a command or program to indicate success (0) or failure (non-zero), used for automated error detection.

\textbf{Graphical User Interface (GUI):} A visual interface that uses icons, menus, and windows for user interaction, contrasted with CLI in this research.

\textbf{Human-Computer Interaction (HCI):} The interdisciplinary field studying how people interact with computers and design technologies for effective human use.

\textbf{Interactive Task Harness:} A testing framework developed for this research that presents tasks to participants and automatically logs their performance data.

\textbf{Large Language Model (LLM):} A type of artificial intelligence model trained on large amounts of text data to understand and generate human-like text, used for command generation and error correction.

\textbf{Likert Scale:} A psychometric scale commonly used in questionnaires to measure attitudes or opinions, employed in this study for user satisfaction ratings.

\textbf{Machine Learning:} A subset of AI that enables systems to automatically learn and improve from experience, underlying the predictive capabilities of AI-assisted CLI tools.

\textbf{Natural Language Processing (NLP):} A branch of AI that helps computers understand, interpret, and manipulate human language, enabling natural language to command translation.

\textbf{Shell:} A command-line interpreter that provides a user interface for accessing operating system services, such as Bash or Zsh.

\textbf{SSH (Secure Shell):} A cryptographic network protocol used for secure remote access to computer systems, utilized in this research for accessing the Docker testing environment.

\textbf{Statistical Power:} The probability that a statistical test will correctly reject a false null hypothesis, considered in the sample size justification for this study.

\textbf{stderr (Standard Error):} A stream used by programs to output error messages, captured in the experimental logging system.

\textbf{stdout (Standard Output):} A stream used by programs to output normal results, monitored in the testing framework for command success validation.

\textbf{Syntax Error:} An error in the structure or grammar of a command that prevents it from being executed correctly, a primary focus of AI-assisted error correction.

\textbf{Task Completion Time:} A key performance metric measuring the time from task presentation to successful completion, used to evaluate CLI efficiency.

\textbf{TypeScript:} A programming language that extends JavaScript with type definitions, used to implement the aiman tool in this research.

\textbf{Unix Philosophy:} A design principle emphasizing small, composable tools that can be combined to solve complex problems, fundamental to CLI design.

\textbf{User Experience (UX):} The overall experience of a person using a product, system, or service, measured through satisfaction questionnaires in this study.

\textbf{Virtual Machine (VM):} A software-based emulation of a computer system, used to provide consistent Linux environments for all participants.

\textbf{Within-Subjects Design:} An experimental design where each participant experiences all conditions, allowing direct comparison while controlling for individual differences.
