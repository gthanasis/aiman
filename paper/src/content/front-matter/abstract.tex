This thesis presents a comprehensive comparative analysis between traditional command-line interfaces (CLI) and AI-assisted CLI systems. The research evaluates the effectiveness of artificial intelligence in enhancing CLI usability by providing intelligent command suggestions, error correction, and contextual guidance.

The study employs a controlled experimental design to measure key performance metrics including task completion time, error rates, learning retention, and user satisfaction. Through systematic user testing with experienced software engineers, this research demonstrates the potential of AI-driven solutions to address the traditional challenges of CLI usage, particularly the steep learning curve and error-prone nature of command-line interactions.

The AI-CLI implementation leverages large language models to provide real-time assistance, offering both proactive command suggestions and reactive error correction. Results indicate significant improvements in user efficiency and reduced learning curves when AI assistance is available, while maintaining the power and flexibility that make CLI tools essential for technical users.

This work contributes to the growing field of AI-assisted computing by providing empirical evidence for the benefits of intelligent CLI interfaces and establishing a framework for evaluating AI-enhanced command-line tools.
