This thesis presents a comprehensive comparative analysis between traditional command-line interfaces (CLI) and aiman, an AI-assisted CLI system. The research evaluates the effectiveness of artificial intelligence in enhancing CLI usability by providing intelligent command suggestions, error correction, and contextual guidance.

The study employs a controlled experimental design to measure key performance metrics including task completion time, success rates, number of attempts, and user satisfaction across ease of use, confidence, and frustration dimensions. Through systematic user testing with experienced software engineers, this research demonstrates the potential of AI-driven solutions to address the traditional challenges of CLI usage, particularly the error-prone nature of command-line interactions.

The aiman implementation leverages large language models to provide real-time assistance, offering reactive error correction and suggestions when commands fail. Results indicate significant improvements in user efficiency and error correction when AI assistance is available, while maintaining the power and flexibility that make CLI tools essential for technical users.

This work contributes to the growing field of AI-assisted computing by providing empirical evidence for the benefits of intelligent CLI interfaces and establishing a framework for evaluating AI-enhanced command-line tools.
